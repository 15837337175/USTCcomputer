% -*- coding: UTF-8 -*-
\documentclass[UTF-8,cs4size]{ctexart}
\title{电磁学小论文 —— 电路分析简介及简易电路分析程序的实现}
\author{PB17000002  古宜民}
\date{2018年6月}
\begin{document} \normalsize
\maketitle
\begin{center}
	摘要
\end{center}

电路是电磁学的重要组成部分,无论在科学研究还是应用实践领域都有必不可少的重大作用,而对电路的分析求解是异常重要的,电路分析有着详细的方法和策略,而计算机辅助在电路分析以及相关复杂领域内有着不可忽视的作用,计算机能极大节省手动计算的时间,并能够应用一些人力不能达到的数值计算和分析算法,但计算机只是工具,其算法还是由理论设计的,如果能理解计算软件内部的算法,那么在使用时就能更加融会贯通。而自己简单实现一个简易的分析程序,则对物理理解和编程能力都有很大锻炼。 


本文将简单介绍电路分析原理,并用C语言实现一个简单的直流电路分析程序。
\clearpage
\section{电路分析简介}
只对在电路分析程序中用到的内容作介绍。 


\subsection{基本概念及定理}
%\subsubsection{元件}
\subsubsection{电路分析常用名词}
\paragraph{支路} 电路中一个或若干个二段元件依次串联且流经的是同一个电流的电路分支称为一个支路。
\paragraph{结点} 电路中三个及以上支路的连接点,称为结点。
\paragraph{回路} 电路中由若干个支路组成的闭合路径,称为回路。
\paragraph{网孔} 平面电路中不含支路的回路,称为网孔。
\subsubsection{基尔霍夫定律}
基尔霍夫定律是电路理论的基石,对电路分析与求解有重大意义。


\paragraph{\textbf{基尔霍夫电流定律(Kirchhoff's Current Law, KCL)}} 电路中的任一结点,在任意时刻,流入该结点的全部支路电流之和等于流出该结点的全部支路电流之和,即
\begin{equation}
	\sum i_{in} = \sum i_{out}
\end{equation}
\paragraph{\textbf{基尔霍夫电压定律(Kirchhoff's Voltage Law, KVL)}} 电路中任一回路,在任意时刻,沿该回路闭合路径绕行一周的全部支路电压的代数和为零,即
\begin{equation}
	\sum_{k=1}^{b}u_k = 0
\end{equation}
其中$b$为回路中包含支路个数,$u_k$为回路中第$k$个支路电压。


由图论知识可以证明,对于具有$n$个结点的电路,其独立的KCL方程有$n-1$个,通常取一个结点为参考结点,其余为独立结点;对于具有$n$个结点,$b$条支路的电路,其独立的KVL方程有$b-n+1$个。
\subsection{计算机方法}
计算机电路分析的方法和思路与平时计算与解题不同,一般解题看中计算的简便性与技巧性,但计算机分析则应看中方法的通用性,即一种模型能解一整类问题,而计算的繁琐则不太重要了。

\section{电路分析程序的实现}
考虑到实现难度及运行效率,用C语言实现。
\end{document}









